\documentclass[a4paper,10pt,oneside]{book}

% packages 
\usepackage{arsclassica}    % fancy layout
\usepackage[english]{babel}\addto{\captionsenglish}{\renewcommand{\bibname}{References}}
\usepackage{caption}         % figure captions
\usepackage[square,numbers,super,sort&compress]{natbib}  % bibliography style
\usepackage[cc]{titlepic}    % enable logo on title page
\usepackage{graphicx}       % logo related

% bibliography
\bibliographystyle{../thesis}

% title setup
\title{ \vspace{3in} Unravelling higher order genome organisation {\small [working
    title]} \\ \vspace{2em} {\large {\bf Introduction}} }
\author{Benjamin L. Moore}
\titlepic{\vspace{2.2in} \includegraphics[width=\textwidth]{/Users/benmoore/hvl/1yrReport/figs/igmm.png}}

\begin{document}

\maketitle

\chapter{Introduction}
\section{Genome organisation}

It's oft-stated that the DNA within each human cell would extend for two metres fully straightened. Instead that same length of DNA packs into a cell nucleus with a diameter in the order of micrometers ($\mu$m).

\subsection{C-methods and Hi-C}

Classical studies of chromosome conformation relied on microscopy techniques to visualise nuclear architecture. These techniques led to the discovery of ``chromosome territories", regions of the nucleus wherein distinct chromosomes were thought to occupy. Finer details of chromatin organisation, such as the proposed 30 nm fibre, were also introduced through microscopy-based techniques. 

With the advent DNA sequencing technology, new experimental methods emerged. Chromosome conformation capture (3C), introduced by Dekker \emph{et al.}\cite{Dekker2002} was the first sequencing-based method of measuring chromosome conformation. The method uses formaldehyde to cross-link nuclear proteins in place, trapping genomic regions that were physically co-located through bound proteins, then to apply a frequent restriction enzyme to shear the sample into fragments. Next, under dilute conditions, DNA fragments are ligated together. The dilute conditions favour ligations between fixed fragments, with the aim of generating hybrid fragments form two genomic regions which were close together in the original preparation. Cross-linking can then be reversed and, in the case of the original 3C method, measured by quantitative PCR using pre-designed primers for your fragments of interest. The end result is a relative measure of interaction frequency between any two regions of interest, in theory directly proportional to their distance in three-dimensional space.

The rapid advancement of sequencing, allowed the original 3C method to be further developed, first through microarray technology, then using high-throughput sequencing. Two protocols were proposed for a 3C-inspired one-to-many assay\cite{Zhao2006, Simonis2006} (both named 4C), whereby interactions were measured for a specific viewpoint fragment against all other restriction fragments genome-wide. The same year a many-to-many assay (5C) allowed measurements for all restriction fragments within a specified region.\cite{Dostie2006} 

The final step was an all-versus-all assay, capable of assaying pairwise interaction frequencies between all restriction fragments of a genome. This assay was published by Lieberman Aiden \emph{et al.}\cite{Lieberman2009} and named Hi-C. The Hi-C method added biotin tagging to pull-down only ligated fragments for sequencing. At the time of publication, resolution of Hi-C data for analysis was limited by sequencing depth, given the huge number of restriction fragments produced by a 6-cutter enzyme (HindIII and NcoI were used in \cite{Lieberman2009}) but the falling costs of sequencing and proven utility of the assay meant subsequent Hi-C papers incrementally increased their sequencing depth, to a point where analysis could be performed at the level of individual restriction fragments, genome-wide.\cite{Dixon2012,  Selvaraj2013a, Jin2013, Rao2014}

\subsection{Hi-C variants}

The interaction maps produced by Hi-C were noticed to exhibit several inherent biases. Fragment properties, such as their length, GC content and mappability, were confounding interaction frequency estimates and therefore needed to be normalised-away before subsequent analysis.\cite{Yaffe2011} A range of statistical techniques were developed to correct for these latent variables\cite{Imakaev2012, Dekker2013, Hu2012, Li2014}

Tethered chromosome capture (TCC)\cite{Kalhor2012} was the first attempt to increase the signal to noise ratio of Hi-C contacts. In this method, ligations take place on a fixed surface, with the aim of preventing spurious ligations between fragments in solution which were not cross-linked.

Hi-C is a population-level assay, as the retrieved interaction counts are from a huge number of different cells. As well as building population-averaged models of genome structure, it is also of interest to probe cell-to-cell variability through single-cell approaches. The first single-cell Hi-C study\cite{Nagano2013} \ldots An obvious limitation of single-cell Hi-C assays is that a single restriction fragment can ligate to at most a single other fragment, meaning even if $100\%$ yield were to be achieved, any $n \times n$ interaction matrix could at most populate $\frac{n}{2}$ cells.

Capture-C is another recent Hi-C derivative which attempts to address resolution problems associated with the genome-wide pairwise assay by enriching for promoter-enhancer interactions using \emph{a priori} selection.\cite{Mifsud2015} It could be said that Capture-C is to Hi-C as exome-capture sequencing is to a whole-genome approach.

In-site Hi-C was a recent refinement of the Hi-C method, from the published of the original method.\cite{Rao2014} The principle difference is that fixation and ligation now happen in place, within intact cell nuclei.

\subsection{Chromosome compartments}

In the paper describing the Hi-C technique,\cite{Lieberman2009}  Lieberman-Aiden \emph{et
  al.} described low-resolution structures they name  ``A'' and ``B'' nuclear compartments. These are regions with a median size of around 5 megabases which showed properties typical
of euchromatin and heterochromatin, respectively. A compartments were observed through 3D-FISH to be centrally-positioned in the nucleus and  ChIP-seq data showed several hallmarks of transcriptional activity. B compartments, conversely, were heterochromatic and lamina-associated regions, with little transcription and repressive histone modifications such as H3k9me3.\cite{Lieberman2009}

Nuclear these compartments were identified through a continuous eigenvector profile, derived from a normalised Hi-C contact matrix.\cite{Lieberman2009} Importantly, this measure holds more information than a simple two-state classification, rather the continuous values can be interpreted as relative levels of compaction or activity.\cite{Dekker2013, Imakaev2012}

\subsection{Topological domains}

A high-profile example of computational analyses leading to new
biological insight can be found in Dixon \emph{et al.}\cite{Dixon2012}
wherein the authors characterised ``topological domains'' (also known
as topological associating domains or TADs), a
megabase-scale feature of genome organisation conserved between human
and mice.

\section{Machine learning in genomics}

The link between epigenomic features and local chromatin state has been
analysed computationally in a number of publications, notably in
developing the
Hidden Markov Model-based ChromHMM\cite{Ernst2012} algorithm which predicts states such as active promoters and enhancers, using a
range of histone marks and other underlying features.\cite{Ernst2011} Similarly a Random Forest-based
algorithm was recently developed to predict enhancers from histone
modification data.\cite{Rajagopal2013} At the opposite end of the
spectrum, theoretical mechanistic models of chromatin folding such as the
``strings and binders switch'' model\cite{Barbieri2012} and the ``fractal
globule'' model\cite{Lieberman2011, Mirny2011, Grosberg1988a} have both produced simulated data
that reflects empirical 3C observations and potentially describe the polymer
dynamics of chromatin folding. However few studies have spanned all of
these levels of chromatin structure and nuclear organisation, and it is not yet known how locus-level chromatin features may
be related to higher order genome organisation. \\

\subsection{ENCODE}

The recent comprehensive ChIP-seq datasets
produced by the ENCODE consortium\cite{Dunham2012} combined with Hi-C
genome-wide contact maps in a number of human cell
types\cite{Dixon2012, Lieberman2011, Kalhor2012} present a remarkable
opportunity to investigate the relationships between local
chromatin features and higher order structure. In this work, a
machine-learning approach was employed to model the
compartmental characteristics of large genomic regions based on their
aggregate levels of various histone marks and DNA binding
proteins. Dissection of the resulting models was then used as a
means of gleaning biological insights into the basis of higher order
structure and of highlighting important differences between cell types.



\begin{small}
\bibliography{/Users/benmoore/Documents/library,/Users/benmoore/Documents/customrefs}
\end{small}

\end{document}
