Recent technological advances have given insights into how genomes are folded in three-dimensions, however many open questions remain about the functional importance of this structure, its variability and its relationships with other features of the genomic and epigenomic landscape. In this work we combine Hi-C datasets describing physical genomic contacts with a large and diverse array of chromatin data derived at a much finer scale, including for example levels of bound transcription factors, histone modifications and expression data. These data are then brought together in a quantitative and rigorous way, through a predictive modelling framework and applied statistical analyses.

First we compare higher order chromatin organisation across a variety of human cell types and find pervasive conservation of chromatin organisation at multiple scales. Beyond this we identify structurally-variable regions that are enhancer-rich and contain loci of known cell-type specific function. We find broad aspects of higher order chromatin organisation, such as chromosome compartments, to be highly predictable in a variety of human cell types. We dissect these models and find them to be generalisable to novel cell types, due to fundamental biological rules linking compartments with key activating and repressive signals. These models describe the strong interconnectedness between locus-level enrichments and depletions of local marks and bound factors with much broader compartmentalisation of large chromosomal regions.

Finally, boundary regions are investigated in terms of chromatin marks and co-localisation with other known nuclear structures. We find boundary complexity to vary between cell types and link TAD aggregations to previously-described lamin-associated domains, as well as exploring the concept of super-boundaries that span multiple levels of organisation. Together these analyses lend evidence to the idea of higher order genome organisation that is largely fixed between cell types, yet one that can selectively vary locally, based on the activation or repression of key loci. 