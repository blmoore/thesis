\documentclass[a4paper,10pt,oneside]{book}

% packages 
\usepackage{arsclassica}    % fancy layout
\usepackage[english]{babel}\addto{\captionsenglish}{\renewcommand{\bibname}{References}}
\usepackage{caption}         % figure captions
\usepackage[square,numbers,super,sort&compress]{natbib}  % bibliography style
\usepackage[cc]{titlepic}    % enable logo on title page
\usepackage{graphicx}       % logo related

% Margins for pretty version ::
%\usepackage[pass]{geometry}
% Margins for university regulations ::
\usepackage[top=2cm, bottom=4cm, left=4cm, right=2.5cm]{geometry}
\usepackage{setspace}
\onehalfspacing

\usepackage{standalone}
\standalonetrue

% don't hang captions
\captionsetup{format=plain}

% bibliography
\bibliographystyle{../thesis}

% title setup
\title{ \vspace{3in} Unravelling higher order genome organisation {\small [working
    title]} \\ \vspace{2em} {\large {\bf Discussion}} }
\author{Benjamin L. Moore}
\titlepic{\vspace{2.2in} \includegraphics[width=\textwidth]{/Users/benmoore/hvl/1yrReport/figs/igmm.png}}

\begin{document}

%\maketitle

\chapter{Discussion}

% General, intro-y type stuff
Since the publication of the Hi-C assay, there has been a flurry of follow-up papers largely focused on improving the technique and/or employing increasingly deep sequencing to achieve higher resolution analyses. A side-effect of this process is a growing bank of chromosome conformation datasets that, following proper reprocessing, can be compared and contrasted to gain biological insights into the importance of higher order chromatin organisation.

Despite the increasing availability of 

Previous studies have compared regions of higher order structure between cell types\cite{Lieberman2009} and species\cite{Dixon2012} but often anecdotally, on a selected region. Here we perform a multi-cell line comparison is a quantitative way, to 


\section{Predictive models of chromosome compartments}

Integrative analyses of locus level chromatin data have allowed the
prediction of functional chromatin states \cite{Ernst2012, Ram2011,
  Dunham2012, Hoffman2013} but these states typically emcompass small
regions such as the enhancers examined here. The prediction of higher
order chromatin domains has received much less attention, and it was
not clear until now that sufficient data existed to allow accurate
predictions. Our data show that accurate predictions of Hi-C derived
eigenvector values, and the nuclear compartment domains based upon
them, are entirely feasible. Strong and significant correlations are
seen between cell types for a variety of human higher order domains,
deliniating variation in replication timing, lamin association and
nuclear compartments derived from Hi-C eigenvectors
\cite{Chambers2013}. The data presented here therefore suggest that a
variety of such domains could be successfully modelled. Given the fact
that the binding patterns of most human chromatin components have not
yet been mapped the models presented here are remarkably successful,
though will undoubtedly improve with further data and algorithm
development. These models also allowed us to probe the features
underlying regions with variable higher order structure between cell
types, revealing enrichments of cell type specific enhancer activity,
and suggesting links between functional chromatin states and higher
order domain dynamics. It is not possible to distinguish cause and
effect using the current data, but it seems likely that the
alterations in domain organization occur prior to enhancer activity.

The recent abundance of epigenomic data in model cell types has
enabled accurate modelling of the transcriptional output of human
promoters, and a rigorously quantitative assessment of the most
influential chromatin features underlying gene expression
\cite{Dong2012}. We have shown that it is possible to construct
comparable models describing the features underlying higher order
chromatin structure, and that their predictive accuracy can be
high. Our analysis exploits Hi-C datasets that have been re-analysed,
from the intitial sequence read mapping onwards, identically for three
different cell types. These data were collated with 35 locus level
ENCODE chromatin datasets, also processed identically, and matched
across the same cell types. In common with previous studies
\cite{Chambers2013, Dixon2012}, we observed good concordance of higher
order chromatin structure, reflected in Hi-C data, between different
cell types. Random forest models summarised the important
relationships among these many variables, providing insights into the
quantitative contributions of locus level chromatin features to higher
order structures. Although certain features were notably more
influential in a particular cell type, the models shared overlapping
constellations of informative features, allowing the cross application
of models between cell types.



The current data suggest that the contributions of certain locus level
chromatin features to higher order structures vary between cell
types. Striking examples include the strong influence of H3K9me3 in
K562 leukemia cells, and EGR1 binding in H1 hESC. EGR1 is a pivotal
regulator of cell fate and mitogenesis with critical roles in
development and cancer \cite{Zwang2012}. While the patterns of
repressive H3K9me3 accumulation have been a focus in the cancer
literature and have been proposed as a diagnostic marker in leukemia
\cite{Muller-Tidow2010}. Similarly, the model for GM12878
(Epstein-Barr virus transformed lymphoblasoid) cells shows a
disproportionate influence of ATF3 binding patterns, and ATF3
induction is a known consequence of virus transformed cells
\cite{Hagmeyer1996}. Thus, the most cell type specific features in
these models may be important indicators of cell type specific
functions. These cell type specific features present a paradox, in
view of the strong correlations in organization genome wide across
different cell types \cite{Chambers2013, Dixon2012}, and the
demonstration that models trained in one cell type often perform well
with data from other cell types. These contradictory observations are
reconciled by the presence of inter-correlated clusters of features
underlying A and B compartments. The shifting membership of these
clusters evidently retains enough similarity between cell types to
enable the cross application of models.

\section{Domain boundaries}

Chromatin domains have been described at multiple scales, from 5 Mb chromosome compartments\cite{Lieberman2009} down to 185 kb contact domains\cite{Rao2014} in human cells. Across all domains, questions remain about how they are constructed and maintained. Two competing ideas are that boundary elements, akin to the classic chromatin insulators, block intra-domain contacts and the spread of heterochromatin and hence create chromatin domains; however, another suggestion is that boundary regions are rather less important and in fact the unavoidable consequence of adjacent self-interacting domains, perhaps instead held together through internal enhancer--promoter interactions, among contacts.  

In favour of functional boundary elements, knockdown of CTCF has been shown to cause increased intraTAD contacts,\cite{Zuin2013} though the same study reported an orthogonal function for cohesin

The incidental boundary hypothesis is supported by data showing that deletion of specific boundary elements in insufficient to cause adjacent domains to merge,(ref XX) In addition, the majority of CTCF sites fall within TADs rather than at their boundaries (approximately $85\%$ of human CTCF sites are non-boundary\cite{Dixon2012}). Further it has been shown that the majority of enhancer--promoter contacts are tissue invariant,\cite{Bouwman2015a} hence these constitutive contacts could account for the high levels of domain conservation reported previously\cite{Lieberman2009, Dixon2012, Chambers2013, Rao2014} and in this work (Chapter \ref{chap:hiccomparison}).

In this work we find an array of chromatin features that, on average, are statistically associated or excluded from TAD or compartment boundaries. Among these are features with a long history of 

% Enrichment of Alu elements suggest CTCF sites inserted, bring about novel loops and domains

As with many biological phenomena the question of whether boundary regions or internal contacts are stabilising chromatin domains is a reductive false dichotomy, and it seems likely that both boundary insulation and interTAD contacts work together to maintain chromatin domains.

Chromatin boundaries, separating TADs and nuclear compartments at
different scales, also showed cell type specific enrichments of
various locus level chromatin features. Across cell types, the
complexity of boundary composition varies considerably so that only a
few features were seen consistently enriched or depleted at
boundaries. Peaks associated with active promoters were notable for
both TAD and compartment boundaries in all cell types. Among the most
influentual variables for the random forest models constructed for the
two hematopoietic cell lines was the ubiquitous transcription factor
YY1, which re-appeared in the analysis of chromatin boundary
regions. Significant enrichments of YY1 were seen at TAD and nuclear
compartment boundaries in all three cell types. Thus, the same protein
was implicated at the level of broad genomic binding patterns (over 1
Mb intervals) and at the level of locally enriched peaks at boundary
regions (spanning 100-500 Kb). This is intriguing as YY1 has recently
been shown to co-localise with the architectural protein
CTCF \cite{Ong2014} and suggests that these proteins cooperate in the
establishment of domain boundaries. The identification of such
features, significantly enriched at boundary regions, provides
potential targets for deletion in experimental studies further
exploring the structure and function of domains
(e.g. \cite{Nora2012}). Both cell type specific and general
constituents of boundaries may have utility in the biomedical
interpretation of genomic variation in noncoding regions of the
genome.

\section{A note on causality}

\section{Insights into higher order chromatin organisation}

% RVS
% largely invariant but with local changes
% functional importance of higher order structure ?

Our results agree with a functional model of genome architecture whereby a majority of the genome is arranged into large static compartments, be they Lamina associated, nucleolus associated or central and accessible chromatin. Indeed, it seems possible that such large, constitutive anchor points may be enough to generate a significant amount of concordance in nuclear architecture between cell types.\cite{Bouwman2015a} This broad overview is  coupled with local changes in different tissues, allowing cell type specific regulation of gene environments through "looping out", detachment from the nuclear lamina and other conceivable mechanisms of variation.

\section{Conclusion}

It has become commonplace to discuss the multi-layered, hierarchical
organization of interphase chromosomes across strata ranging from
nuclear compartments, down to the spectra of histone modifications and
bound proteins at individual sub-genic regions. However we lack a
detailed understanding of how these strata interact. We have shown
that our perspectives of features occurring at different strata can be
bridged by modelling approaches, and the models produced can used to
explore the interrelationships between these different features
quantitatively. 

We constructed cell type specific models of nuclear
organization, as reflected in Hi-C derived eigenvector profiles, to
discover the most influential features underlying higher order
structures. We found open and closed compartments to be
well-correlated with combinatorial patterns of histone modifications
and DNA binding proteins, enabling accurate predictive models. These
models could be cross-applied successfully between cell types
highlighting constellations of common structural features associated
with different nuclear compartments as expected. Dissection of the
most influential variables also revealed important differences between
models, consistent with the known biological contrasts among these
cell types, such as the prominence of EGR1 in embryonic stem cells and
H3K9me3 in the leukaemia cell line. Investigation of regions showing
variable nuclear organization across the three cell types under study,
revealed enrichments for cell type specific enhancer activity, often
nucleated at genes with known roles in cell type specific
functions. Finally we used model predictions to examine boundary
composition between higher order domains across cell types. Among
enrichments of a large number of factors observed at different
boundaries in different cell types, CTCF and YY1 were found
consistently and may cooperate to establish domain boundaries. In
summary, we show that integrative modelling of large chromatin dataset
collections using random forests can generate useful insights into
chromosome structure and seed testable hypotheses for further
experimental studies.

\section{Future research}

\ifstandalone
\begin{small}
\bibliography{/Users/benmoore/Documents/library,/Users/benmoore/Documents/customrefs}
\end{small}
\fi

\end{document}
