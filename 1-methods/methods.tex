\documentclass[a4paper,10pt,oneside]{book}

% packages 
\usepackage{arsclassica}    % fancy layout
\usepackage[english]{babel}\addto{\captionsenglish}{\renewcommand{\bibname}{References}}
\usepackage{caption}         % figure captions
\usepackage[square,numbers,super,sort&compress]{natbib}  % bibliography style
\usepackage[cc]{titlepic}    % enable logo on title page
\usepackage{graphicx}       % logo related

% bibliography
\bibliographystyle{../thesis}

% title setup
\title{ \vspace{3in} Unravelling higher order genome organisation {\small [working
    title]} \\ \vspace{2em} {\large {\bf Methods section}} }
\author{Benjamin L. Moore}
\titlepic{\vspace{2.2in} \includegraphics[width=\textwidth]{/Users/benmoore/hvl/1yrReport/figs/igmm.png}}

\begin{document}

\maketitle

\chapter{Methods}

\section{Input data}\label{input-data}

Some more stuff here.

\subsection{Hi-C data}\label{hi-c-data}

Raw Hi-C reads were downloaded from three published datasets through
GEO\citep{Barrett2013} or the SRA\citep{Leinonen2011a} with identifiers:
GSE35156 (H1 hESC), GSE18199 (K562) and SRX030113 (GM12878). These
paired reads were mapped independently to theand mapped to the genome
(hg19/GRCh37). Mapping was performed using the \texttt{hiclib} software
package\citep{Imakaev2012} and \texttt{bowtie2}\citep{Langmead2012} with
the \texttt{-{}-very-sensitive} flag. Mapped reads were then binned into
contact maps and iteratively corrected\citep{Imakaev2012}. The
\texttt{hiclib} software was also used for eigenvector expansion of each
intrachromosomal contact map, performed independently on each chromosome
arm.

\subsection{Locus-level features}\label{locus-level-features}

Genome-wide ChIP-seq datasets for: 22 DNA binding proteins and 10
histone marks were made available by the ENCODE
consortium\citep{Dunham2012, Boyle2014} along with DNase I
hypersensitivity and H2A.z occupancy, for each of the Tier 1 ENCODE cell
lines used in this work: H1 hESC, K562 and GM12878. These data were
pre-processed using MACSv2\citep{Zhang2008} to produce fold-change
relative to input chromatin. GC content was also calculated and used in
the featureset.

\subsection{Clustering input features}

To quantify collinearity of input features, correlation matrices built from genome-wide vectors of input feature measures were build and hierarchicaly clustered. The "significance" of observed clustering was assessed using sub- and super-sampled bootstrapping, with stable clusters deemed significant. The \texttt{pvclust} R package

\section{Modelling}\label{modelling}

\subsection{Random Forest}\label{sec:rf}

Random Forest regression\citep{Breiman2001a} was used as implemented in
the R package \texttt{randomForest}.\citep{Liaw2002} Parameters of
$mtry = \frac{n}{3} = 12$ and $ntrees = 200$ were assumed as they
approximate the defaults and are known to be largely
insensitive.\citep{Hastie2001}

Variable importance within Random Forest regression models was measured
using mean decrease in accuracy in the out-of-bag (OOB) sample. This
represents the average difference (over the forest) between the accuracy
of a tree with permuted and unpermuted versions of a given variable, in
units of mean squared error (MSE).\citep{Cutler2007}

\subsection{Model performance}\label{model-performance}

The effectiveness of the modelling approach was measured by four
different metrics. Prediction accuracy was assessed by the Pearson
correlation coefficient between the predicted and observed eigenvectors
(determined by 10-fold cross-validation), and the root mean-squared
error (RMSE) of the same data. Classification error, when predictions
where thresholded into $A \geq 0; B < 0$, was also calculated using
accuracy (\% correct classifications or True Positives) and area under
the receiver operating characteristic (AUROC) curve. Together these give
a comprehensive overview of the model performance, both in terms of
regression accuracy of the continuous eigenvector, and in how that same
model could be used to label discrete chromatin compartments.

For cross-application of cell type specific models, a single Random
Forest regression model was learned from all 1 Mb bins for a given cell
type. This was then used to predict all bins from each of the other two
cell types.

\subsection{Other modelling approaches}
 
Linear regression was used as a baseline for comparison with more complicated approaches such as Random Forest. If the same modelling accuracy could be achieved with simple multiple linear regression, this would be a faster and more interpretable modelling framework.

Partial least squares regression was also used to model compartment profiles. This method is well-suited to highly correlated inputs.

\section{Variable regions}\label{variable-regions}

\subsection{Stratification by
variability}\label{stratification-by-variability}

Median absolute deviation (MAD) was chosen as a robust measure of the
variability in a given 1 Mb block between the three primary cell types
used in this work: H1, K562 and GM12878. Blocks were ranked by this
measure and split into thirds that represented ``low'' variability (the
third of blocks with the lowest MAD), ``mid'' and ``high'' variability.
Each subgroup was then independently modelled using the
previously-described Random Forest approach.

``Flipped'' regions are those whose compartment state differs in one
cell type relative to the other two. For example, if a 1 Mb bin was
classified as ``open'' in H1 hESC and ``closed'' in both K562 and
GM12878, this is said to be a ``flipped'' compartment (to open).

\subsection{Enhancer enrichment}\label{enhancer-enrichment}

Enhancer annotations were collected from the ChromHMM / SegWay combined
annotations in each cell type.\citep{Hoffman2013} Enhancers were
considered ``shared'' if there was an overlapping enhancer annotation in
either of the two other cell types, and labelled as ``tissue-specific''
otherwise.

This was repeated for other chromatin states.

\section{Boundaries}\label{boundaries}

\subsection{TADs}\label{tads}

TAD boundaries were called using the software provided in
\citet{Dixon2012} using their recommended parameters. For the generation
of boundary profiles, the same parameters were used: input features were
averaged into 40 kb bins spanning $\pm500$ kb from the boundary centre.

To align boundaries between cells \ldots

\subsection{Compartments}\label{sec:compartments}

Compartment boundaries were called by first training a two-state hidden
Markov model (HMM) on the compartment eigenvector and then using the
Viterbi algorithm to predict the most likely state sequence that
produced the observed values. The point at which transitions occurred
between states was taken as a boundary which was then extended $\pm 1.5$
Mb to give a 3 Mb window in which a boundary was though to occur.

To test for the enrichment or depletion of a chromatin feature over a
given boundary, a two tailed Mann-Whitney test was used to compare the
boundary bin with the ten outermost bins of the window (5 from either
side). The significance level at $\alpha = 0.01$ was then
Bonferonni-adjusted for multiple testing correction, and results with
\emph{p}-values exceeding this threshold were deemed significantly
enriched or depleted at a given boundary.

\subsection{MetaTADs}\label{sec:m-metatad}

\section{Giemsa band comparison}\label{giemsa-band-comparison}

Cytogenic band data and Giemsa stain results were downloaded from the
UCSC genome browser (table \texttt{cytoBandIdeo}). The genomic
co-ordinates are an approximation of cytogenic band data inferred from a
large number of FISH experiments.\citep{Furey2003}

To compare G-band boundaries with our compartment data, we allowed for a
$\pm 500$ kb inaccuracy in G-band boundary. For each G-band boundary,
the minimum absolute distance to any compartment or TAD boundary was
calculated for each cell type. To generate a null model, \ldots


\begin{small}
\bibliography{/Users/benmoore/Documents/library,/Users/benmoore/Documents/customrefs}
\end{small}

\end{document}
